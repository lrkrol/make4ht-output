\documentclass[a4paper]{article}

\usepackage[utf8]{inputenc}
\usepackage[margin=1.25in]{geometry}
\usepackage{graphicx}
\usepackage[bottom]{footmisc}
\usepackage[unnumberedbib,notocbib]{apacite}
\usepackage{authblk}
\usepackage{amsmath}

% possessive reference
\newcommand\citeApos[1]{\citeauthor{#1}'s (\citeyearNP{#1})}

% footer to indicate publication status
% \usepackage{fancyhdr} \pagestyle{fancy} \renewcommand{\headrulewidth}{0pt}
% \lhead{} \rhead{} \cfoot{\tiny This is the author's version of the work.}

% local bibliography
\begin{filecontents}{localbib.bib}
@INPROCEEDINGS{7727294,  author={Xue-Qin Huo and Zheng, Wei-Long and Lu, Bao-Liang},  booktitle={2016 International Joint Conference on Neural Networks (IJCNN)},   title={Driving fatigue detection with fusion of EEG and forehead EOG},   year={2016},  volume={},  number={},  pages={897-904},  doi={10.1109/IJCNN.2016.7727294}}

@article{LIANG2019105,
title = {Prediction of drowsiness events in night shift workers during morning driving},
journal = {Accident Analysis \& Prevention},
volume = {126},
pages = {105-114},
year = {2019},
note = {10th International Conference on Managing Fatigue: Managing Fatigue to Improve Safety, Wellness, and Effectiveness”.},
issn = {0001-4575},
doi = {10.1016/j.aap.2017.11.004},
author = {Yulan Liang and William J. Horrey and Mark E. Howard and Michael L. Lee and Clare Anderson and Michael S. Shreeve and Conor S. O’Brien and Charles A. Czeisler},
}

@article{Aric__2018,
	doi = {10.1088/1361-6579/aad57e},
	year = 2018,
	month = {aug},
	publisher = {{IOP} Publishing},
	volume = {39},
	number = {8},
	pages = {08TR02},
	author = {P Aric{\`{o}} and G Borghini and G Di Flumeri and N Sciaraffa and F Babiloni},
	title = {Passive {BCI} beyond the lab: current trends and future directions},
	journal = {Physiological Measurement}
}

@INPROCEEDINGS{8448945,  author={Park, Corey and Shahrdar, Shervin and Nojoumian, Mehrdad},  booktitle={2018 IEEE 10th Sensor Array and Multichannel Signal Processing Workshop (SAM)},   title={EEG-Based Classification of Emotional State Using an Autonomous Vehicle Simulator},   year={2018},  volume={},  number={},  pages={297-300},  doi={10.1109/SAM.2018.8448945}}
\end{filecontents}

\begin{document}

\title{}
\date{}

\author[1,*]{Laurens R. Krol}
\author[2]{Thorsten O. Zander}
\affil[1]{Neuroadaptive Human-Computer Interaction, Brandenburg University of Technology, Cottbus-Senftenberg, Germany}
\affil[2]{Zander Laboratories B.V., Amsterdam, The Netherlands} 
\affil[*]{Correspondence: \texttt{lrkrol@gmail.com}}


\maketitle
% \thispagestyle{fancy}

\begin{abstract}

\end{abstract}


\section*{1.1.3}

A brain-computer interface (BCI) is a neurotechnological system that allows human mental states to be directly accessed by a computer. To that end, the BCI system reads human brain activity, most commonly using electroencephalography (EEG), applies a number of signal processing and feature extraction steps, and employs machine learning algorithms in order to classify the resulting features as being indicative of certain mental states. This was first demonstrated in the 1970s \cite{vidal1977realtimedetect} has been under continuous development ever since, primarily in medical contexts. Here, BCI systems are envisioned to, for example, enable paralysed or otherwise motor-impaired patients to regain independence by replacing natural neural pathways with BCIs \cite{wolpaw2002commcontrol}. However, more than that can be done with technology that has direct access to human mental states. \citeA{zander2011passive} in particular introduced the concept of \emph{passive} BCIs, where technology uses arbitrary, naturally occurring mental states as input, as opposed to those \emph{active} BCIs that are used for intentional, direct communication and control \cite{krol2018interactivity}. This latter category simply replaces existing means of control with a BCI. The former, however, enables new forms of \emph{implicit} human-computer interaction, through which technology can automatically adapt to its human user, without this user actively communicating anything, or even being aware of any communication taking place \cite{krol2022definingnat}.



\section*{1.1.4}

A number of studies utilising BCI systems in automotive contexts have relied on active BCIs, thus replacing the driver's manual control over certain systems with intentional BCI-based control.

For example, 
...

Aside from that, many studies have used EEG and other modalities to investigate states like fatigue and workload during driving \cite<e.g.,>{lin2005drowsiness,7727294}. These and other findings have been used to inform automotive adaptation using passive BCI. For example, \citeA{kohlmorgen2007realtimeworkload} demonstrated how a passive BCI system can detect mental load during driving and automatically adjust secondary tasks to better suit the driver's current state. \citeA{haufe2014realworldbraking} demonstrated the feasibility of using EEG activity to passively detect emergency braking intention. \citeA{LIANG2019105} developed a real-time, pBCI-based drowsiness detector to allow accident prevention measures to be taken when drivers are detected to be incapable of attentive driving. See i.a. \citeA{Aric__2018} for a further discussion of the literature.

% The above findings are limited to fairly specific mental states. The proposed adaptation of ADAS parameters to more general neural correlates of user experience 

In the current context, a final mention is made of an exploratory study attempting to extract emotional responses to different behaviour patterns of self-driving cars \cite{8448945}. This exploratory, $n=1$ study used simple measures from a single electrode with correspondingly inconclusive results, but noted parietal alpha/beta power ratio as a candidate indicator.


\section*{Eigene Vorarbeiten}

Approximately a decade ago, the concept of passive BCI was formally introduced by one of the consortial partners \cite{zander2011passive}. Since then, various mental states have been investigated by this group for use with passive BCI, including loss of control \cite{}, workload \cite{krol2016workload}, predictive coding and error-related perceptions \cite{zander2016nat}, and subjective interpretations of positive/negative behaviour \cite{krol2019saliencevalence}. 

In-vehicle physiological recordings are prone to both environmental and internal artefacts, due to the car's electronic equipment and the natural movement of the participant. We have previously investigated the art and nature of these automotive artefact sources as well as EEG hardware susceptibility to typical in-vehicle human movements \cite{zander2017dry}. Subsequently, in a separate study, we investigated ability of state-of-the-art signal processing algorithms to clean data recorded during real driving of 15 participants \cite{krol2017cleaning}. Results show that, taking the identified artefactual sources into account, neural correlates can be recovered to allow pBCI classification. Furthermore, in the context of adaptive cruise control (ACC) behaviour, we have found that, aside from neurophysiological correlates, heart rate and blink duration may serve as an indicator of ACC-related driving experience \cite{brouwer2017physdriving}.

Aside from cars, passive BCI systems have also been applied in full-body flight simulators. In particular, it has been demonstrated that passive BCI can be used to detect pilot perceptions of cockpit alerts in order to inform a cognitive model \cite{klaproth2020cognitivemodel}. The passive BCI system used for this was context-independent, being trained on an abstract task and subsequently transferred onto realistic data recorded in the flight simulator \cite{krol2022pilots}. As such, this approach is not limited to flight, but can be used in cars as well.



% \begin{figure}[h]
%     \centering
%     \label{fig:label}
%     \includegraphics[width=\textwidth]{figs/erpimages.png}
%     \caption{Caption}
% \end{figure}

% \begin{table}
%     \centering
%     \label{tab:label}
%     \begin{tabular}{ccc}
%         1 & 2 & 3 \\
%     \end{tabular}
%     \caption{Caption}
% \end{table}



% \section*{Acknowledgements}

% Part of this work was supported by the Deutsche Forschungsgemeinschaft (ZA 821/3-1). 


\bibliographystyle{apacite} 
\bibliography{bibliography,localbib}
\end{document}
